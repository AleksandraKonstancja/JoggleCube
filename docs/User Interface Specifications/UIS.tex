\documentclass{project}
\usepackage[pdfauthor={Nathan Williams, Alex Thaumwood},pdftitle={Software Engineering Group Project, User Interface Specification},pdftex]{hyperref}
\begin{document}
\title{Software Engineering Group Project}
\subtitle{User Interface Specification, Use Case Document}
\author{Nathan Williams - naw21, Alex Thaumwood - alt38}     
\shorttitle{Use Case Document}
\version{0.1.1}
\status{Draft}
\date{2018-02-01}
\configref{SE-N66-TEST}
\maketitle
\tableofcontents
\newpage
\section{INTRODUCTION}

\subsection{Purpose of this Document}

This document describes the main use cases of the system. It should be read in the context of the Group Project, taking into account the JoggleCube Requirements Specification \cite{SE.QA.CSRS}

\subsection{Scope}

This document covers who the typical users of the system are, their needs, use cases and any errors they may come across.

The document should be read by the developers working on implementing the system.

\subsection{Objectives}
This Document aims to:
\begin{itemize}
	\item define who the users of the system are.
	\item identify their specific needs.
	\item explain the use cases of the system for each type of user.
	\item identify possible error conditions and what is to be done about them.
\end{itemize}

\section{Typical Users}
\subsection{Second Year Computer Science Students}
	As described in the JoggleCube Requirements Specification \cite{SE.QA.CSRS}, these users are familiar with standard software tools, and with WIMP software. They are by default, quite lazy, and so the software should provide the indicated features with the fewest possible mouse movements and keystrokes.
	
	

\section{Use Cases}
	\subsection{Second Year Computer Science Students}
	Play against a new Grid
	Compete against a saved grid
	View highscores
\section{Error Conditions}


\clearpage
\addcontentsline{toc}{section}{REFERENCES}
\begin{thebibliography}{5}
\bibitem{SE.QA.CSRS} \emph{Software Engineering Group Projects}
JoggleCube Game Requirements Specification.
C. J. Price SE.QA.CSRS. 1.0 Release.
\end{thebibliography}
\clearpage
\addcontentsline{toc}{section}{DOCUMENT HISTORY}
\section*{DOCUMENT HISTORY}
\begin{tabular}{|l | l | l | l | l |}
\hline
Version & CCF No. & Date & Changes made to Document & Changed by \\
\hline
0.1 & N/A & 2018-02-01 & Initial creation & NAW21 \\
\hline
0.1.1 & N/A & 2018-02-02 & Begun writing the introduction and Typical Users Section & NAW21 \\
\hline
\end{tabular}
\label{thelastpage}
\end{document}